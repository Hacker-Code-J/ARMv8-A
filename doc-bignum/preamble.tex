% Preamble with packages, custom commands, etc.
\usepackage{commath, amsmath, amsthm}
\usepackage{polynomial}
\usepackage{enumerate}
\usepackage{soul} % highlight
\usepackage{lipsum}  % Just for generating text

% Theorem
\newtheorem{axiom}{Axiom}[chapter]
\newtheorem{theorem}{Theorem}[chapter]
\newtheorem{proposition}[theorem]{Proposition}
\newtheorem{corollary}{Corollary}[theorem]
\newtheorem{lemma}[theorem]{Lemma}

\theoremstyle{definition}
\newtheorem{definition}{Definition}[chapter]
\newtheorem{remark}{Remark}[chapter]
\newtheorem{exercise}{Exercise}[chapter]
\newtheorem{example}{Example}[chapter]
\newtheorem*{note}{Note}

% Colors
\usepackage[dvipsnames,table]{xcolor}
\definecolor{titleblue}{RGB}{0,53,128}
\definecolor{chaptergray}{RGB}{140,140,140}
\definecolor{sectiongray}{RGB}{180,180,180}

\definecolor{thmcolor}{RGB}{231, 76, 60}
\definecolor{defcolor}{RGB}{52, 152, 219}
\definecolor{lemcolor}{RGB}{155, 89, 182}
\definecolor{corcolor}{RGB}{46, 204, 113}
\definecolor{procolor}{RGB}{241, 196, 15}
\definecolor{execolor}{RGB}{90, 128, 127}

\definecolor{comments}{HTML}{6A9955} % A kind of forest green.
\definecolor{keyword}{HTML}{569CD6} % A medium blue..
\definecolor{string}{HTML}{CE9178} % A reddish-brown or terra cotta.
\definecolor{function}{HTML}{DCDCAA} % A beige or light khaki.
\definecolor{number}{HTML}{B5CEA8} % A muted green.
\definecolor{type}{HTML}{4EC9B0} %  A turquoise or teal.
\definecolor{class}{HTML}{4EC9B0} % Similar to types, a turquoise or teal.

% Define custom colors
\definecolor{commentColor}{rgb}{0.25,0.5,0.35}
\definecolor{keywordColor}{rgb}{0.6,0.1,0.1}
\definecolor{stringColor}{rgb}{0.5,0.1,0.5}
\definecolor{backColor}{rgb}{0,0,0} % Black background

\definecolor{aesblue}{RGB}{30,144,255}
\definecolor{aesgreen}{RGB}{0,128,0}
\definecolor{aesred}{RGB}{255,69,0}
\definecolor{aesgray}{RGB}{112,128,144}

% Fonts
\usepackage[T1]{fontenc}
\usepackage[utf8]{inputenc}
\usepackage{newpxtext,newpxmath}
\usepackage{sectsty}
\allsectionsfont{\sffamily\color{titleblue}\mdseries}

% Page layout
\usepackage{geometry}
\geometry{a4paper,left=.8in,right=.6in,top=.75in,bottom=1in,heightrounded}
\usepackage{fancyhdr}
\fancyhf{}
\fancyhead[LE,RO]{\thepage}
\fancyhead[LO]{\nouppercase{\rightmark}}
\fancyhead[RE]{\nouppercase{\leftmark}}
\renewcommand{\headrulewidth}{0.5pt}
\renewcommand{\footrulewidth}{0pt}

% Chapter formatting
\usepackage{titlesec}
\titleformat{\chapter}[display]
{\normalfont\sffamily\Huge\bfseries\color{titleblue}}{\chaptertitlename\ \thechapter}{20pt}{\Huge}
\titleformat{\section}
{\normalfont\sffamily\Large\bfseries\color{titleblue!100!gray}}{\thesection}{1em}{}
\titleformat{\subsection}
{\normalfont\sffamily\large\bfseries\color{titleblue!75!gray}}{\thesubsection}{1em}{}

% Table of contents formatting
\usepackage{tocloft}
\renewcommand{\cftchapfont}{\sffamily\color{titleblue}\bfseries}
\renewcommand{\cftsecfont}{\sffamily\color{chaptergray}}
\renewcommand{\cftsubsecfont}{\sffamily\color{sectiongray}}
\renewcommand{\cftchapleader}{\cftdotfill{\cftdotsep}}

% TikZ
\usepackage{tikz}
\usepackage{tikz-cd}
\usetikzlibrary{math} % for calculations
\usetikzlibrary{matrix, positioning, arrows.meta, calc, shapes.geometric, shapes.multipart, chains}
\usetikzlibrary{decorations.pathreplacing,calligraphy}

\usepackage{pgfplots}
\usepgfplotslibrary{fillbetween}
\pgfplotsset{compat=1.15}

% Pseudo-code
\usepackage[linesnumbered,ruled]{algorithm2e}
\usepackage{algpseudocode}
\usepackage{setspace}
\SetKwComment{Comment}{/* }{ */}
\SetKwProg{Fn}{Function}{:}{end}
\SetKw{End}{end}
\SetKw{DownTo}{downto}

% Define a new environment for algorithms without line numbers
\newenvironment{algorithm2}[1][]{
	% Save the current state of the algorithm counter
	\newcounter{tempCounter}
	\setcounter{tempCounter}{\value{algocf}}
	% Redefine the algorithm numbering (remove prefix)
	\renewcommand{\thealgocf}{}
	\begin{algorithm}
	}{
	\end{algorithm}
	% Restore the algorithm counter state
	\setcounter{algocf}{\value{tempCounter}}
}

%Tcolorbox
\usepackage{tcolorbox}
\tcbset{colback=white, arc=5pt}

\definecolor{axiomcolor}{HTML}{a88bfa}
\definecolor{defcolor}{RGB}{52, 152, 219}
\definecolor{procolor}{RGB}{241, 196, 15}
\definecolor{thmcolor}{RGB}{231, 76, 60}
\definecolor{lemcolor}{RGB}{155, 89, 182}
\definecolor{corcolor}{RGB}{46, 204, 113}
\definecolor{execolor}{RGB}{90, 128, 127}

% Define a new command for the custom tcolorbox
\newcommand{\axiombox}[2][]{%
	\begin{tcolorbox}[colframe=axiomcolor, title={\color{white}\bfseries #1}]
		#2
	\end{tcolorbox}
}

\newcommand{\defbox}[2][]{%
	\begin{tcolorbox}[colframe=defcolor, title={\color{white}\bfseries #1}]
		#2
	\end{tcolorbox}
}

\newcommand{\lembox}[2][]{%
	\begin{tcolorbox}[colframe=lemcolor, title={\color{white}\bfseries #1}]
		#2
	\end{tcolorbox}
}

\newcommand{\probox}[2][]{%
	\begin{tcolorbox}[colframe=procolor, title={\color{white}\bfseries #1}]
		#2
	\end{tcolorbox}
}

\newcommand{\thmbox}[2][]{%
	\begin{tcolorbox}[colframe=thmcolor, title={\color{white}\bfseries #1}]
		#2
	\end{tcolorbox}
}

\newcommand{\corbox}[2][]{%
	\begin{tcolorbox}[colframe=corcolor, title={\color{white}\bfseries #1}]
		#2
	\end{tcolorbox}
}



% Listings
\usepackage{listings}
\renewcommand{\lstlistingname}{Code}%
\lstdefinestyle{C}{
	language=C,
	backgroundcolor=\color{white},
	basicstyle=\ttfamily\color{black},
	commentstyle=\color{green!70!black},
	keywordstyle={\bfseries\color{purple}},
	keywordstyle=[2]{\bfseries\color{red}},
	keywordstyle=[3]{\bfseries\color{type}},
	keywordstyle=[4]{\bfseries\color{JungleGreen}},
	keywordstyle=[5]{\bfseries\color{Magenta}},
	keywordstyle=[6]{\bfseries\color{RoyalBlue}},
	keywordstyle=[7]{\bfseries\color{Turquoise}},
	otherkeywords={bool, inline, size\_t, FILE},
	morekeywords=[2]{;},
	morekeywords=[3]{i8, i32, i64, u8, u32, u64, field, word},
	morekeywords=[4]{
		rdtsc, measure\_cycle, measure\_time, stringToWord
	},
	morekeywords=[5]{
		main
	},
	morekeywords=[6]{false, true, strlen, sscanf, printf},
	morekeywords=[7]{
		ONE, SIZE, IS\_32\_BIT\_ENV, PRIME, PRIME\_INVERSE
	},
	numberstyle=\tiny\color{gray},
	stringstyle=\color{purple},
	showstringspaces=false,
	breaklines=true,
	frame=single,
	framesep=3pt,
	%frameround=tttt,
	framexleftmargin=3pt,
	numbers=left,
	numberstyle=\small\color{gray},
	xleftmargin=15pt, % Increase the left margin
	xrightmargin=5pt,
	tabsize=4,
	belowskip=0pt,
	aboveskip=4pt
}
\lstdefinestyle{zsh}{
	language=bash,                  % Set the language to bash (closest to Zsh)
	backgroundcolor=\color{backColor},
	commentstyle=\color{commentColor}\ttfamily,
	keywordstyle=\color{keywordColor}\bfseries,
	stringstyle=\color{stringColor}\ttfamily,
	showspaces=false,               % Don't show spaces as underscores
	showstringspaces=false,         % Don't highlight spaces in strings
	breaklines=true,                % Automatic line breaking
	frame=none,                     % No frame around the code
	basicstyle=\ttfamily\color{white}, % White basic text color for contrast
	extendedchars=true,             % Allow extended characters
	captionpos=b,                   % Caption-position at bottom
	escapeinside={\%*}{*)},         % Allow LaTeX inside your code
	morekeywords={echo,ls,cd,pwd,exit,clear,man,unalias,zsh,source}, % Add more keywords
	upquote=true,                   % Ensure straight quotes are used
	literate={\$}{{\textcolor{red}{\$}}}1
	{:}{{\textcolor{red}{:}}}1
	{~}{{\textcolor{red}{\textasciitilde}}}1, % Color certain characters
}
\lstdefinestyle{normal}{
	basicstyle=\ttfamily\small,
	linewidth=\textwidth,  % Sets the listing width to match the text width
	frame=single, % Border
	basicstyle=\tt\scriptsize, % Code font style
	breaklines=true, % Break long lines into multiple lines automatically
	aboveskip=2\baselineskip, % Vertical whitespace before the listing
	tabsize=4, % How many space to a tab
}
% Usage: \lstinputlisting[style=zsh]{sourcefile.sh}
% or \begin{lstlisting}[style=zsh] ... \end{lstlisting}


% Table
\usepackage{booktabs}
\usepackage{tabularx}
\usepackage{multicol}
\usepackage{multirow}
\usepackage{array}
\usepackage{longtable}
\newcolumntype{C}[1]{>{\centering\arraybackslash}p{#1}}

% Hyperlinks
\usepackage{hyperref}
\hypersetup{
	colorlinks=true,
	linkcolor=titleblue,
	filecolor=black,      
	urlcolor=titleblue,
}

%Ceiling and Floor Function
\usepackage{mathtools}
\DeclarePairedDelimiter{\ceil}{\lceil}{\rceil}
\DeclarePairedDelimiter{\floor}{\lfloor}{\rfloor}


\newcommand{\mathcolorbox}[2]{\colorbox{#1}{$\displaystyle #2$}}